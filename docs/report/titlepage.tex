\begin{titlepage}
\begin{flushleft}
\textbf{\LARGE{Researcher Survey of Replication in Geography}}
\vspace{60pt}
\setlength{\columnsep}{20pt}

\noindent \textbf{Peter Kedron}, Corresponding Author\\
  School of Geographical Sciences and Urban Planning, Arizona State University\\
  pkedron@asu.edu\\
  ORCID iD: 0000-0002-1093-3416\\
  \hfill\break
   \textbf{Joseph Holler}\\
  Department of Geography, Middlebury College\\
  josephh@middlebury.edu\\
  ORCID iD: 0000-0002-2381-2699\\
  \hfill\break
  \textbf{Sarah Bardin}\\
  School of Geographical Sciences and Urban Planning, Arizona State University\\
  sfbardin@asu.edu\\
  ORCID iD: 0000-0001-8657-1725 \\
  \hfill\break

\vspace{20pt}

\renewcommand*\abstract{\flushleft\large\textbf{Abstract}\hfill}
\begin{abstract}
\begin{flushleft}
    \noindent While the number of reproduction and replication studies undertaken in the social and behavioural sciences continues to rise, such studies have not yet become commonplace in geography. 
    The small number of studies that have attempted to reproduce geographic research suggest that many studies cannot be fully reproduced, or are simply missing components needed to attempt a reproduction. 
    Despite this suggestive evidence, we have not yet systematically assessed the use of reproducible research practices across the discipline's diverse research traditions, or identified what factors have barred geographers from conducting more reproduction studies.
    This pre-analysis plan outlines the procedures and methods we will use to assess the current use of reproducible research practices in geography and identify barriers to conducting reproductions. 
    This document was prepared after the development of the sampling frame and survey instrument, but before the beginning of data collection.
    \end{flushleft}
\end{abstract}

\vspace{40pt}

Submission date: May 05, 2022
\end{flushleft}
\end{titlepage}
